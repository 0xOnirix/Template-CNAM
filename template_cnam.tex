% Template LaTeX CNAM
% Auteur : Adrien 'Onirix' Ketterer
% Créé le : 09/10/2024
% Modifié le : 02/06/2025
%
% Ce template prend en compte les exigences de mise en forme du CNAM décrites
% dans le "Guide pour la rédaction du mémoire d'ingénieur CNAM", disponible ici : https://ecole-ingenieur.cnam.fr/medias/fichier/guide-redaction-memoire-vf-2023_1695841357339-pdf
%
% Soit: 
% - Police : Times New Roman
% - Taille police : 12
% - Interligne : 1.5
%
%
% Le template utilise des variables (dans la section "Variables") pour la page de garde.

\documentclass[12pt]{article}

\usepackage{fullpage}       % Page de garde améliorée
\usepackage{mathptmx}       % Utilise la police Times New Roman
\usepackage{graphicx}       % Permet d'avoir des images
\usepackage{setspace}       % Permet de modifier l'espacement (interligne, etc)
\usepackage{fancyhdr}       % Pour personnaliser les en-têtes et pieds de page
\usepackage{tabularx}       % Permet de mieux gérer les tableaux
\usepackage{hyperref}       % Permet d'avoir des hyperliens
\usepackage{ifthen}         % Permet d'avoir des conditions if, then et else
\usepackage{float}
\usepackage{xurl}           % Permet de "casser" les url trop longues (pas de débordement)

\hypersetup{                    % parametrage des hyperliens (lié à hyperref)
    colorlinks=true,                % colorise les liens
    breaklinks=true,                % permet les retours à la ligne pour les liens trop longs
    urlcolor= blue,                 % couleur des hyperliens
    linkcolor= black,               % couleur des liens internes aux documents (index, figures, tableaux, equations,...)
    citecolor= green                % couleur des liens vers les references bibliographiques
}

\renewcommand{\headrulewidth}{0pt}                  % Enlève la barre horizontale en haut
\renewcommand{\contentsname}{Table des matières}    % Change le nom du sommaire
\renewcommand{\listfigurename}{Liste des figures}   % Change le nom de la liste des figure
\renewcommand{\listtablename}{Liste des tableaux}   % Change le nom de la liste des tableaux
\renewcommand{\arraystretch}{1.5}                   % Change le padding des cases des tableaux, Valeur par défaut : 1, ici on augmente à 1.5
\renewcommand{\tablename}{Tableau}                  % Change le début du caption des tableaux
\onehalfspacing                                     % Définit un interligne de 1.5 pour tout le document

%%%%%%%%%%%%%%% Variables %%%%%%%%%%%%%%%%%
\author{NOM AUTEUR}

\newcommand\obtention{Diplôme d'ingénieur cybersécurité CNAM}          % Ce qu'il y a à obtenir, ex: le diplôme d'Ingénieur CNAM, L'UE XXX

\newcommand\specialiteDiplome{Informatique}                             % Spécialité du diplôme

\newcommand\optionDiplome{Cybersécurité}                                % Option du diplôme

\newcommand\titre{Mémoire de fin de cycle Ingénieur cybersécurité}      % Sujet/titre du document

\newcommand\dateSoutenance{xx/xx/xxxx}

\newcommand\presidentJury{Président}

\newcommand\membresJury{
    Jury1 - Fonction

    Jury2 - Fonction
}

%%%%%%%%%%%%%%%%%%%%%%%%%%%%%%%%%%%%%%%%%%%
\date{\today}
\makeatletter

\begin{document}
\sloppy                         % Permet une meilleure gestion des overflow sur la droite
\pagestyle{empty}               % Supprime temporairement les numéros de page

%%%%%%%%%%%%% Page de garde %%%%%%%%%%%%%%%
\begin{titlepage}

    \begin{figure}
        \raggedleft\includegraphics[scale=0.3]{images/logo_CNAM.png}
   \end{figure}

    \vspace*{0.5cm}

    \enlargethispage{3cm}
    \begin{center}
 
        
        CONSERVATOIRE NATIONAL DES ARTS ET METIERS\\[0.5cm]
        CENTRE CNAM RÉGIONAL DE BRETAGNE\\[0.5cm]

        \vspace*{0.5cm}
        \centerline{\hbox to 0.5\textwidth{\hrulefill}}     % Ligne horizontale centrée à 50% de la largeur
        \vspace*{1cm}
        
        MEMOIRE
        \vspace*{0.5cm}

        présenté en vue d'obtenir\\
        \vspace*{0.5cm}

        \obtention\\
        \vspace*{0.5cm}

        SPECIALITE : \specialiteDiplome\\
        \vspace*{0.5cm}


        OPTION : \optionDiplome\\
        \vspace*{0.5cm}


        par\\
        \vspace*{0.5cm}

        \@author
        \vspace*{0.5cm}

        \centerline{\hbox to 0.5\textwidth{\hrulefill}}
        \vspace*{1cm}

        \titre\\
        \vspace*{0.5cm}


        Soutenu le \dateSoutenance\\
        \vspace*{0.5cm}


    \end{center}


    \vfill{}                                    % "Repousse" le texte en bas - https://texnique.fr/osqa/questions/555/comment-repousser-le-texte-en-bas-de-page
    {\parindent0pt
        JURY\\
        PRESIDENT : \presidentJury - Président(e)\\
        MEMBRES : 
        
        \membresJury

    }

\end{titlepage}

\newpage
\vspace*{\fill} % Page blanche
%%%%%%%%%%%%% Table des matières %%%%%%%%%%%%%%%
\newpage

\tableofcontents

%%%%%%%%%%%%% Table des figures %%%%%%%%%%%%%%%%
\newpage

\listoffigures

%%%%%%%%%%%%%%%%%% Document %%%%%%%%%%%%%%%%%%%%
\newpage

\setcounter{page}{1}    % Réinitialise le compteur de pages à 1
\pagestyle{plain}       % Réactive les numéros de page à partir d'ici

\pagestyle{fancy}       % Ce bloc met le numéro de page en bas à droite
\fancyhf{}
\fancyfoot[R]{\thepage}

\section*{Liste des abréviations}
\addcontentsline{toc}{section}{Liste des abréviations}

\textbf{Exemple :} Exemple\\

\newpage

\section*{Glossaire}
\addcontentsline{toc}{section}{Glossaire}

\textbf{Exemple (ex) :} Exemple\\

\newpage 

\section*{Introduction}
\addcontentsline{toc}{section}{Introduction}

\newpage

\section{Section 1}
\subsection{Subsection 1}

\newpage

\section{Conclusion}

\newpage

\begingroup
\setlength{\parindent}{0pt}  % Désactive l'indentation

\section*{Bibliographie}
\addcontentsline{toc}{section}{Bibliographie}

\endgroup  % Fin de la désactivation de l'indentation

\newpage

%%%%%%%%%%%%%%%%%%% Quatrième de couverture %%%%%%%%%%%%%%%%%%%

\centerline{\hbox to 0.5\textwidth{\hrulefill}}

\begin{center}

\section*{RESUME}

\end{center}

Ceci est un résumé

\textbf{Mots clés:} latex, template, CNAM\\

\centerline{\hbox to 0.5\textwidth{\hrulefill}}

\begin{center}

\center{\section*{ABSTRACT}}

\end{center}

This is an abstract

\textbf{Keywords:} latex, template, CNAM

\end{document}