% Template LaTeX CNAM
% Auteur : Adrien 'Onirix' Ketterer
% Créé le : 09/10/2024
% Modifié le : 04/06/2025
%
% Ce template prend en compte les exigences de mise en forme du CNAM décrites
% dans le "Guide pour la rédaction du mémoire d'ingénieur CNAM", disponible ici : https://ecole-ingenieur.cnam.fr/medias/fichier/guide-redaction-memoire-vf-2023_1695841357339-pdf
%
% Soit: 
% - Police : Times New Roman
% - Taille police : 12
% - Interligne : 1.5
%
% Le template utilise des variables (dans la section "Variables") pour la page de garde.
%
% Attention, il faut le compiler avec xelatex ! Bien lire le README

\documentclass[12pt]{article}

\usepackage[absolute,overlay]{textpos}  % Permet le centrage absolue d'éléments
\usepackage[acronym]{glossaries}        % Ajoute les acronymes
\usepackage{fontspec}                   % Gère les polices d'écriture
\usepackage{fullpage}                   % Page de garde améliorée
\usepackage{mathptmx}                   % Utilise la police Times New Roman
\usepackage{graphicx}                   % Permet d'avoir des images
\usepackage{setspace}                   % Permet de modifier l'espacement (interligne, etc)
\usepackage{fancyhdr}                   % Pour personnaliser les en-têtes et pieds de page
\usepackage{tabularx}                   % Permet de mieux gérer les tableaux
\usepackage{hyperref}                   % Permet d'avoir des hyperliens
\usepackage{eso-pic}                    % Peremt de rajouter quelque chose sur chaque page
\usepackage{ifthen}                     % Permet d'avoir des conditions if, then et else
\usepackage{float}
\usepackage{xurl}                       % Permet de "casser" les url trop longues (pas de débordement)

\setmainfont{Times New Roman}           % Définit la police sur Times New Roman

\hypersetup{                            % parametrage des hyperliens (lié à hyperref)
    colorlinks=true,                            % colorise les liens
    breaklinks=true,                            % permet les retours à la ligne pour les liens trop longs
    urlcolor= blue,                             % couleur des hyperliens
    linkcolor= black,                           % couleur des liens internes aux documents (index, figures, tableaux, equations,...)
    citecolor= blue                             % couleur des liens vers les references bibliographiques
}

\renewcommand{\headrulewidth}{0pt}                  % Enlève la barre horizontale en haut
\renewcommand{\contentsname}{Table des matières}    % Change le nom du sommaire
\renewcommand{\listfigurename}{Liste des figures}   % Change le nom de la liste des figure
\renewcommand{\listtablename}{Liste des tableaux}   % Change le nom de la liste des tableaux
\renewcommand{\arraystretch}{1.5}                   % Change le padding des cases des tableaux, Valeur par défaut : 1, ici on augmente à 1.5
\renewcommand{\tablename}{Tableau}                  % Change le début du caption des tableaux
\renewcommand{\glossaryname}{Glossaire}             % Renomme le glossaire
\renewcommand{\refname}{Bibliographie}              % Renomme la bibliographie
\onehalfspacing                                     % Définit un interligne de 1.5 pour tout le document
\newboolean{showPageNumberFooter}                   % Création d'une variable pour afficher ou non le numéro de page
\setboolean{showPageNumberFooter}{false}            % Désactivé par défaut
\pagestyle{empty}                                   % Supprime la numéroation de page par défaut

%%%%%%%%%%%%%%% Variables %%%%%%%%%%%%%%%%%
\author{NOM AUTEUR}

\newcommand\obtention{Diplôme d'ingénieur cybersécurité CNAM}          % Ce qu'il y a à obtenir, ex: le diplôme d'Ingénieur CNAM, L'UE XXX

\newcommand\specialiteDiplome{Informatique}                             % Spécialité du diplôme

\newcommand\optionDiplome{Cybersécurité}                                % Option du diplôme

\newcommand\titre{Mémoire de fin de cycle Ingénieur cybersécurité}      % Sujet/titre du document

\newcommand\dateSoutenance{xx/xx/xxxx}

\newcommand\presidentJury{Président}

\newcommand\membresJury{
    NomJury1 - Fonction

    NomJury2 - Fonction
}

\newcommand\piedDePage{Pied de page}

\makeglossaries

%%%%%%%%%%%%%%% Acronymes %%%%%%%%%%%%%%%%%

\newacronym{pentest}{Pentest}{Penetration Testing - Test d'intrusion}

%%%%%%%%%%%%%%% Glossaire %%%%%%%%%%%%%%%%%

\newglossaryentry{forensique}
{
    name=forensique,
    description={ou investigation numérique, est un domaine scientifique  dans lequel on cherche à récupérer et analyser des supports numériques potentiellement suspicieux\cite{forensiquedef}}
}

%%%%%%%%%%%%%%%%%%%%%%%%%%%%%%%%%%%%%%%%%%%

% Ajoute le pied de page sur chaque page (page de garde exclu) + numéro de page
\AddToShipoutPictureBG{
  \begin{textblock*}{\dimexpr\paperwidth - 2cm}(1cm,27cm)
    \small\itshape
    \makebox[\dimexpr\paperwidth - 2cm][s]{%
      \piedDePage  \hfill \ifthenelse{\boolean{showPageNumberFooter}}{\the\numexpr\value{page} + 1\relax}{}
    }
  \end{textblock*}
}

\date{\today}
\makeatletter

\begin{document}
\sloppy                         % Permet une meilleure gestion des overflow sur la droite

%%%%%%%%%%%%% Page de garde %%%%%%%%%%%%%%%
\begin{titlepage}

    \begin{figure}
        \raggedleft\includegraphics[scale=0.3]{images/logo_CNAM.png}
   \end{figure}

    \vspace*{0.5cm}

    \enlargethispage{3cm}
    \begin{center}
 
        
        CONSERVATOIRE NATIONAL DES ARTS ET METIERS\\[0.5cm]
        CENTRE CNAM RÉGIONAL DE BRETAGNE\\[0.5cm]

        \vspace*{0.5cm}
        \centerline{\hbox to 0.5\textwidth{\hrulefill}}     % Ligne horizontale centrée à 50% de la largeur
        \vspace*{1cm}
        
        MEMOIRE
        \vspace*{0.5cm}

        présenté en vue d'obtenir\\
        \vspace*{0.5cm}

        \obtention\\
        \vspace*{0.5cm}

        SPECIALITE : \specialiteDiplome\\
        \vspace*{0.5cm}


        OPTION : \optionDiplome\\
        \vspace*{0.5cm}


        par\\
        \vspace*{0.5cm}

        \@author
        \vspace*{0.5cm}

        \centerline{\hbox to 0.5\textwidth{\hrulefill}}
        \vspace*{1cm}

        \titre\\
        \vspace*{0.5cm}


        Soutenu le \dateSoutenance\\
        \vspace*{0.5cm}


    \end{center}

    {\parindent0pt
        JURY\\
        PRESIDENT : \presidentJury - Président(e)\\
        MEMBRES : 
        
        \membresJury

    }

    % Position absolue du pied de page : 27cm depuis le haut (ajuste selon besoin)
    \begin{textblock*}{\textwidth}(0cm,27cm)
        {\small \textit{\piedDePage}}
    \end{textblock*}


\end{titlepage}
\clearpage
\setcounter{page}{2}
\thispagestyle{empty}
\null

%%%%%%%%%%%%% Remerciements %%%%%%%%%%%%%%%
\newpage
\addcontentsline{toc}{section}{Remerciements}
\section*{Remerciements}

%%%%%%%%%%%%% Table des matières %%%%%%%%%%%%%%%
\newpage
\tableofcontents

\setboolean{showPageNumberFooter}{true} % Réactive la numérotation des pages

%%%%%%%%%%%%% Table des figures %%%%%%%%%%%%%%%%
\newpage
\addcontentsline{toc}{section}{Liste des figures}
\listoffigures

%%%%%%%%%%%%%%%%%% Acronymes %%%%%%%%%%%%%%%%%%%%
\newpage

\addcontentsline{toc}{section}{Liste des acronymes}
\printglossary[type=\acronymtype, title=Liste des acronymes]

%%%%%%%%%%%%%%%%%% Glossaire %%%%%%%%%%%%%%%%%%%%
\newpage

\glsaddall

\printglossary[toctitle=Glossaire]
\addcontentsline{toc}{section}{Glossaire}

%%%%%%%%%%%%%%%%%% Document %%%%%%%%%%%%%%%%%%%%
\newpage
\section*{Introduction}
\addcontentsline{toc}{section}{Introduction}

Exemple pour l'entrée acronyme \gls{pentest} et l'entrée glossaire \gls{forensique}.

\newpage

\section{Section 1}
\subsection{Subsection 1}

\newpage

\section{Conclusion}

\newpage

\section*{Annexes}
\addcontentsline{toc}{section}{Annexes}

\newpage

\addcontentsline{toc}{section}{Bibliographie}
\bibliographystyle{plain}
\bibliography{biblio}

%%%%%%%%%%%%%%%%%%% Quatrième de couverture %%%%%%%%%%%%%%%%%%%
\newpage
\centerline{\hbox to 0.5\textwidth{\hrulefill}}

\begin{center}

\section*{RESUME}
\addcontentsline{toc}{section}{RESUME}

\end{center}

Ceci est un résumé

\textbf{Mots clés:} latex, template, CNAM\\

\centerline{\hbox to 0.5\textwidth{\hrulefill}}

\begin{center}

\center{\section*{ABSTRACT}}
\addcontentsline{toc}{section}{ABSTRACT}

\end{center}

This is an abstract

\textbf{Keywords:} latex, template, CNAM

\end{document}